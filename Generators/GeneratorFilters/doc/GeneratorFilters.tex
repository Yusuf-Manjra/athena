\documentclass[11pt]{article}
\newdimen\SaveHeight \SaveHeight=\textheight
\textwidth=6.5in
\textheight=8.9in
\textwidth=6.5in
\textheight=9.0in
\hoffset=-.5in
\voffset=-1in
\def\topfraction{1.}
\def\textfraction{0.}   
\def\topfraction{1.}
\def\textfraction{0.}           
\begin{document}
\title{GeneratorFilters: A package for generator event filtering in  Athena\\
Version in release 11.0.0 and later}
\author{ Ian Hinchliffe (I\_Hinchliffe@lbl.gov)}
%\date{Nov, 2002}
\maketitle       
 
This package contains a set of algorithms that run from within Athena.
The algorithms extract the generated Monte Carlo event from the
transient event store (Storegate) and then test for various criteria.
The base class. { \bf GenFilter.cxx} extracts the event from Storegate
and makes it available; all the other filter classes inherit from this.


The events are in HepMC format which is independent of the actual
Monte Carlo generator. The same filter can therefore be used with
Isajet, Herwig, Pythia {\it etc}. Uses are directed to the 
ElectronFilter.cxx which illustrates the basic logic and can be used
as a template for preparing other classes. (Note that SampleFilter.cxx
is easier to read as it does not use the base class, {\bf  users
should not use SampleFilter as a template as their code will not then
be protected against changes in the way the events are held in
Storegate}.

First some properties settable from the jobOptions are declared and
defaults set. {\bf Note that the parameters are in ATLAS units, the
  energies and momenta are in MeV}

\begin{verbatim}
ElectronFilter::ElectronFilter(const std::string& name, 
      ISvcLocator* pSvcLocator): GenFilter(name,pSvcLocator) {
  //----------------------------    
  declareProperty("Ptcut",m_Ptmin = 10000.0); \\ 10 GeV 
  declareProperty("Etacut",m_EtaRange = 3.0); 
\end{verbatim}

The initialization and loading of the event is done by the base
class. The core of the algorithm is in 

\begin{verbatim}
StatusCode ElectronFilter::filterEvent() {}
\end{verbatim}

The event is accessed and each particle examined

\begin{verbatim}
// Loop over all events in McEventCollection
  McEventCollection::iterator itr;
  for (itr = m_cCollptr->begin(); itr!=m_cCollptr->end(); ++itr) {
    // Loop over all particles in the event
    HepMC::GenEvent* genEvt = (*itr)->pGenEvt();
    for(HepMC::GenEvent::particle_iterator pitr=genEvt->particles_begin();
        pitr!=genEvt->particles_end(); ++pitr ){
\end{verbatim}
{\bf pitr} is an iterator for the current particle. We examine its PDG
Id (see the Review of Particle Properties for these codes) and its
transverse momentum and rapidity. Note that since HepMC's GenParticle
inherits from the the CLHEP particle class, the  transverse momentum
and rapidity are available.

\begin{verbatim}
    for(HepMC::GenEvent::particle_iterator pitr=genEvt->particles_begin();
        pitr!=genEvt->particles_end(); ++pitr ){
      if( ((*pitr)->pdg_id() == 11) || ((*pitr)->pdg_id() == -11) ){
        if( ((*pitr)->momentum().perp() >=m_Ptmin) && 
            std::abs((*pitr)->momentum().pseudoRapidity()) <=m_EtaRange){
          return StatusCode::SUCCESS;
        }
\end{verbatim}
If the particle is an electron (or positron) and the   transverse momentum
and rapidity are in the accepted range, we have found an electron that
passes the requirement. Hence we return. The default is
setFilterPassed(true) so we have passed the filter. If the loop runs
with out finding a candidate then we fall to the end of the loop

\begin{verbatim}
        }
      }
    }   
  }
  setFilterPassed(false);
  return StatusCode::SUCCESS;
\end{verbatim}

The code is set so that the filter failed.

The existing filters are as follows.

\begin{itemize}
\item ElectronFilter: looks for electron or positron above some pt and
inside some rapidity range. Set properties via
\begin{verbatim}
ElectronFilter.Etacut = 2.7
ElectronFilter.Ptcut = 4000. \\4 GeV
\end{verbatim}
\item ZtoLeptonFilter: looks for a Z that decays to $e^+e^-$ or
  $\mu^+\mu^-$. This was used in DC0
\item MultiLeptonFilter: looks for $e$ or $\mu$ of either sign. 
 Set properties via
\begin{verbatim}
MultiLeptonFilter.NLeptons = 4
MultiLeptonFilter.Etacut = 2.7
MultiLeptonFilter.Ptcut = 4000. \\4 GeV
\end{verbatim}
\item JetFilter: Looks for either cone or grid jets.
\begin{verbatim}
JetFilter = Algorithm( "JetFilter" )
JetFilter.JetNumber = 1
JetFilter.EtaRange = 2.7
JetFilter.JetThreshold = 17000.;  # Note this is 17 GeV
JetFilter.GridSizeEta=2; # sets the number of (approx 0.06 size) eta
JetFilter.GridSizePhi=2; # sets the number of (approx 0.06 size) phi cells
#the above pair are not used if cone is selected
JetFilter.JetType=FALSE; #true is a cone, false is a grid
JetFilter.ConeSize=0.4; #cone size (not used if grid is selected)
\end{verbatim}
\item PhotonFilter
\begin{verbatim}
PhotonFilter.NPhotons = 2
PhotonFilter.Etacut = 2.5
PhotonFilter.Ptcut = 10000. \\4 GeV
\end{verbatim}
\item BSignalFilter. The interested user should read the documentation of the Generators/PythiaB
  package to understand the functionality of this filter.
\end{itemize}


The filters are used in an Athena seqence as in the following example
\begin{verbatim}
theApp.Dlls  += [ "Herwig_i" ]
theApp.Dlls  += [ "TruthExamples" ]
theApp.Dlls += [ "HbookCnv" ]
theApp.Dlls += [ "GeneratorFilters" ]
theApp.Dlls += ["GaudiAlg"]
theApp.HistogramPersistency = "HBOOK"
theApp.TopAlg = ["Sequencer/Generator"]
Generator.Members = ["Herwig", "ElectronFilter", "HistSample"]
HistogramPersistencySvc.OutputFile  = "herwig.hbook";
\end{verbatim}
If the ElectronFilter passes then HistSample is executed and the event
ends up in the histogram. If it fails then HistSample is not
executed. A more complicated example showing how to write out
filtered events can be found in
Generators/GeneratorFilters/share/JobOptions.pythia.higgs.filter.py


The base class keeps account of the number of passed and failed events and reports this at output. You can stop the processing when a userdefined number of events has passed. This is done via the commmand by setting the variable
TotalPassed. For example,

\begin{verbatim}
ElectronFilter.TotalPassed=1200
\end{verbatim}

will stop cause the athena to stop once 1200 event have passed the filter. To use this the number of requested events must be big enough so that at least TotalPassed will occur.


\end{document}

