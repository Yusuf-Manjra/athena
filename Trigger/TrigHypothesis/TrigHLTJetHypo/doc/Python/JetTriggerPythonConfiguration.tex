\documentclass{beamer}

% \usepackage{beamerthemesplit} // Activate for custom appearance

\title{The Trigger Jet Hypo - Python Configuration}
\author{P Sherwood}
\date{10 November 2020}

\addtobeamertemplate{navigation symbols}{}{%
    \usebeamerfont{footline}%
    \usebeamercolor[fg]{footline}%
    \hspace{1em}%
    \insertframenumber/\inserttotalframenumber
}
\usepackage{xcolor}
\usepackage{caption}
\usepackage{subcaption}
\usepackage{wrapfig}


\begin{document}

\frame{\titlepage}
% \begin{frame}
%       \frametitle{Outline}
%       \begin{quotation}
%       My mind is clearer now - at last all too well I can see where we all soon will be
%	\end{quotation}
%       \tableofcontents[pausesections]
%\end{frame}



%\subsection {Tree Structures}
%\subsection{HypoTrees}
%\subsection{ConditionTrees}
%\subsection{Dynamic Programming}

%\part{The Jet Hypo as a Tree}
%\frame{\partpage
%	
%%{
%  \frametitle{Part 1: Fast Reduction Generalities}
%  }
%}

\frame 
 {
 \frametitle{Context}
 
 \begin{block}{High level: C++}
 At initialisation, the Trigger Framework:
 \begin{itemize}
\item instantiates 1 JetHypoTool per chain.
\item instantiates 1 JetHypoAlgorithm
\item passes the tools to the Algorithm
\end{itemize}
\end{block}

\begin{block}{Role of the python configuration}
Provide the initialisation for any JetHypoAlgTool and its helper AlgTool classes
\end{block}
}


\frame 
 {
 \frametitle{Jet Hypo AlgTool structure (C++)}
   \begin{block}{}
   %\begin{wrapfigure}{l}{0.82\linewidth}
      \begin{minipage}[T]{0.75\linewidth}
   \includegraphics[trim = 20mm 130mm 0mm 25mm, clip, width= 0.98\linewidth]{Pictures/class_overview}
   \end{minipage}%
   % \end{wrapfigure}   AlgTools to be intialised in yellow
         \begin{minipage}[T]{0.25\linewidth}
         \begin{small}
          {\bf Yellow:} AlgTools to be initialised by jet~trig python\\
          {\it C++ \& python classes are isomorphic}
          
          \hrulefill\\
          {\bf Red:} Non-component objects instantiated by HelperConfig
          \end{small}
           \end{minipage}
   \end{block}   
   \begin{block}{}
   Experience from Run 2:  different hypo approaches can be encapsulated in the Matcher (FastReduction, AlgTrees, Maxflow...)
   \end{block}
}

\frame 
 {
 \frametitle{Overview of the jet trigger python configuration}
 \begin{block}{Goal}
 Create python instances of TrigJetHypoTool and associated classes in order to initialise the  corresponding C++ classes
  \end{block}
  \begin{block}{Classes to be initialised for each chain}
  \begin{itemize}
  \item TrigJetHypoTool
  \item TrigJetHypoToolHelper
  \item TrigJetHypoToolHelperConfig\_fastreduction
  \item multiple TrigJetConditionConfig
  \end{itemize}


  \end{block}
  
}

\frame{
	\frametitle{How do we do it?}
	
	\begin{block}{View from 35 000 ft}
	
		\begin{enumerate}
			\item Trigger framework sends a chain dict.
			\item chain dict $\rightarrow$ chain label (hierarchical tree representation)
			\item chain label $\rightarrow$ node tree  (hierarchical tree representation)
			\item process node tree to recover python AlgTools
		\end	{enumerate}
	\end{block}
}

\begin{frame}[fragile]
	\frametitle{Chain Label examples}

	\begin{block}{chain label examples: scenario = `simple'}
		\begin{itemize}
			\item 'simple([(38et, 0eta320)])',
			\item 'simple([(38et, 0eta320)(40et, 0eta320)])',
			\item 'simple([(38et, 0eta320, 011jvt)])',
		\end{itemize}
	\end{block}
	
	\begin{block}{chain label examples: scenario $\neq$ `simple', nested scenarios}
		and([]\\
    \makebox[1cm]{}{simple([(30et,500neta)(30et,peta500)])}\\
    \makebox[1cm]{}combgen([(10et,0eta320)]\\
	    \makebox[2cm]{}dijet([(34djmass,26djdphi)])\\
	    \makebox[2cm]{}simple([(10et,0eta320)(20et,0eta320)]\\
	    \makebox[1cm]{})\\
    )
	\end{block}
\end{frame}


\frame{

\frametitle{chainDict $\rightarrow$ chain label (chanDict2jetLabel.py)}
% \frametitle{What's the deal with the `simple' scenario?}
\begin{block}{hypo scenarios}
`hypoScenario'  is a key in the chainDict. Its values decides how the chain label is constructed.
\end{block}

\begin{block}{hypoScenario = `simple'}
\begin{itemize}
\item The `simple' scenario maintains continuity with Run 2 chains.
\item The chain names and chainDict remain as before remain as in Run 2.
\item The jet trig python code assembles the chain label from various entries in the chainDict.
\end{itemize}
\end{block}

\begin{block}{otherwise...}
\begin{itemize}
\item The `hypoScenario' key in the chain dict contains the chain label 
which is taken from the chain name. {\it ex:}~HLT\_j0\_{\color{red}vbenf}{\color{blue}SEP30etSEP34mass35SEP50fbet}\_L1J20
\item scenario ({\color{red}vbenf}) identifies a string template which is filled in using {\color{blue}parameters} in hypoScenario.
\end{itemize}
\end{block}

}

\frame{
	\frametitle{chain label $\rightarrow$ node tree}
	\begin{block}{chain label vs node treel}
	\begin{center}
		\begin{tabular}{|l|c|c|} \hline
		& ChainLabel & node tree \\ \hline
		type & string & node objects \\
		par-child &nested parens & child elements \\
		extendable & no & typically via visitors \\
		\hline
		\end{tabular}
	\end{center}

%	\begin{itemize}
%		\item chain label is structured - legal chain labels conform to a syntax
%		\item  the syntax allows for the prescription of a tree through nested parentheses.
%	\item chain label contains information on Conditions (cut predicates)
%	\end{itemize}
	\end{block}
\begin{block}{Why have a node tree as well?}
Chain label and Node tree: same tree structure. Nodes allow python mechanics to set up AlgTools
\end{block}
	
\begin{block}{Node tree creation and evolution}
\begin{itemize}
\item JetTrigParser reads the chain label. Syntax is checked. Creates initial node tree.
\item Attributes of the nodes modifiable (are python objects!)
\item Code organisation: encapsulate node modification actions in visitors - avoid complex Node class.
\end{itemize}

\end{block}
}

\frame{
	\frametitle{ChainLabel $\rightarrow$ node tree: Generalities}
	\begin{block}{chain label}
	\begin{itemize}
		\item chain label is structured - legal chain labels conform to a syntax
		\item  the syntax allows for the prescription of a tree through nested parentheses.
	\item chain label contains information on Conditions (cut vars, window cuts) and child nodes.
	\end{itemize}
	\end{block}

\begin{block}{conversion to a node tree}
\begin{itemize}
\item node tree is tree made of node instances, Each node contains data, and a list of child nodes.
\item the node tree has the strupwdcture prescribed by the chain label.
\item the conversion from chain label to node tree is done using Jet
\end{itemize}

\end{block}
}

\frame{
\frametitle{ChainLabel Syntax}

Alphabet: $\Sigma=$\{[a-z][0-9],\}.
\begin{center}
\begin{tabular}{lll}
{\it lower} & $\rightarrow$ & $a|b|c|d|e|f|g|h|i|j|k|l|m|n|o|p|q|r|s|t|u|v|w|x|y|z$\\
{\it digit \it} & $\rightarrow$ &$0|1|2|3|4|5|6|7|8|9$\\
{\it parR}& $\rightarrow$ & ) \\
{\it parL }  & $\rightarrow$ &( \\
{\it braL} & $\rightarrow$ &[ \\
{\it braR \it} & $\rightarrow$ &] \\
{\it comma \it} & $\rightarrow$ &, \\
{\it scenario \it}& $\rightarrow$ & {\it name, parL, windowList, \{scenario\}, parR }\\
{\it name \it} & $\rightarrow$ &{\it lower,\{lower\}} \\
{\it windowList \it} & $\rightarrow$ & {\it braL window, \{comma, window\}, braR \it}\\
{\it window \it} & $\rightarrow$ &{\it \{digit\}, lower, \{lower\}, \{digit\}}  \\
{\it scenarioList \it} & $\rightarrow$ &{\color{red}{\it scenario} $|$ {\it scenario scenarioList} } \\
{\it  label$_S$ \it} & $\rightarrow$ &{\it scenarioList}  \\
\end{tabular}
\end{center}

The scenarioList is what allows nesting (recursion).
}

\frame{
	\frametitle{ChainLabel $\rightarrow$ node tree: Parsing}
%	\begin{block}{}
%	\begin{itemize}
%		\item The parser checks the ChainLabel obeys the syntax (else ERROR)
%		\item Creates an isomorphic node tree
%	\end{itemize}
%	\end{block}
	
	\begin{block}{}
	   	\begin{minipage}[T]{0.5\linewidth}
	 	\includegraphics[trim = 30mm 0mm 30mm 30mm, clip, width= \linewidth]{Pictures/hypoStateMachine}
		\end{minipage}%
		\begin{minipage}[T]{0.5\linewidth}
		The parser ensures correct ChainLabel syntax (else Error Condition), node tree ismorphic to the ChainLabel.
		\end{minipage}
	\end{block}
	
}	

\frame{
 \frametitle{Recursive Classes = Trees}
 \begin{block}{UML representations of a tree}
  \includegraphics[trim = 10mm 230mm 0mm 20mm, clip, width= 1.0\linewidth]{Pictures/recursive}
\end{block}

 \begin{block}{Visitors traverse Trees}
   \begin{minipage}[T]{0.55\linewidth}
   \includegraphics[trim = 10mm 195mm 60mm 25mm, clip, width= 1.0\linewidth]{Pictures/visitor}
   \end{minipage}%
    \begin{minipage}[T]{0.45\linewidth}
  	A: navigation\\Visitor: functionality.\\\mbox{}\\
	New visitors can be introduced with modifying A.
   \end{minipage}
 \end{block}

 }
 
  
\frame
{
  \frametitle{The visitors, and what they do}
 % \begin{minipage}[T]{0.7\linewidth}
\begin{block}{HLT\_j260\_320eta490\_L1J75\_31ETA49}
 \includegraphics[trim = 10mm 180mm 0mm 20mm, clip, width= 0.8\linewidth]{Pictures/node}

 \begin{description}
\item[TreeParameterExpander\_visitor] Modifies  node: attach a conf\_attr list of dictionaries
\end{description}
\end{block}
\begin{block}{}
 \begin{description}
 \item[ConditionsToolSetterFastReduction\_visitor] Modifies self: attaches  a TrigJetHypoToolHelper instance
 \end{description}
 \end{block}
 }
 
 \frame
{
  \frametitle{Python TrigJetHypoTool creation: Sequence Overview}
  \begin{block}{}
   \includegraphics[trim = 0mm 0mm 0mm 0mm, clip, width= \linewidth]{Pictures/Config_seq}
   \end{block}
 }
 
  
 \frame
{
  \frametitle{Python TrigJetHypoTool creation: TreeExpander\_Visitor}
  \begin{block}{}
   \includegraphics[trim = 0mm 0mm 0mm 0mm, clip, width= \linewidth]{Pictures/TreeParameterExpanderVisitor_seq}
   \end{block}
 }
 
  \frame
{
  \frametitle{Python TrigJetHypoTool creation: TreeToolSetterFastReduction Visitor}
  \begin{block}{}
         \begin{minipage}[T]{0.45\linewidth}
	   \includegraphics[trim = 0mm 0mm 100mm 0mm, clip, width= \linewidth]{Pictures/ConditionsToolSetterFastreduction_seq}
	 \end{minipage}%
	 \begin{minipage}[T]{0.55\linewidth}
	 Non-leaf nodes $\rightarrow$ AcceptAll nodes\\
	 Collect information from tree to:
	 \begin{itemize}
		\item build tree vector
		\item instantiate ConditionMakers
	\end{itemize}
	 \end{minipage}

   \end{block}
 }

 

 \end{document}

\frame{
 \frametitle{Concepts}
 \begin{block}{TrigJetHypoToolHelper}
 \begin{description}
\item[Tree] A graph where each node has a single parent, except the rroot node. Each node as zero or more children.
\item [Jet Group] a vector of jets
\item[Condition] A predicate that accepts a jet group.
\item[Grouper] A device that splits jet groups into smaller jet groups
\item[Matcher] A device that presents jet froups to Conditions
\end{description}
 \end{block}
  \begin{block}{FastReduction}
  \begin{description}
\item[ Condition Tree] Expresses hierarchical relations among Conditions
\item[FastReducer] Class that performs FastReduction 
\end{description}
\end{block}
 }

